\documentclass[conference]{IEEEtran}
\IEEEoverridecommandlockouts
% The preceding line is only needed to identify funding in the first footnote. If that is unneeded, please comment it out.
\usepackage{cite}
\usepackage{amsmath,amssymb,amsfonts}
\usepackage{algorithmic}
\usepackage{graphicx}
\usepackage{textcomp}
\usepackage{xcolor}
\usepackage{booktabs}   % Para linhas de tabela profissionais
\usepackage{siunitx}    % Para o alinhamento decimal (coluna S)
\def\BibTeX{{\rm B\kern-.05em{\sc i\kern-.025em b}\kern-.08em
    T\kern-.1667em\lower.7ex\hbox{E}\kern-.125emX}}
\begin{document}
% Comando para criar cabeçalhos de tabela com quebra de linha
\newcommand{\thead}[1]{\begin{tabular}{@{}c@{}}#1\end{tabular}}

\title{Conference Paper Title*\\
{\footnotesize \textsuperscript{*}Note: Sub-titles are not captured in Xplore and
should not be used}
\thanks{Identify applicable funding agency here. If none, delete this.}
}

\author{\IEEEauthorblockN{1\textsuperscript{st} Given Name Surname}
\IEEEauthorblockA{\textit{dept. name of organization (of Aff.)} \\
\textit{name of organization (of Aff.)}\\
City, Country \\
email address or ORCID}
\and
\IEEEauthorblockN{2\textsuperscript{nd} Given Name Surname}
\IEEEauthorblockA{\textit{dept. name of organization (of Aff.)} \\
\textit{name of organization (of Aff.)}\\
City, Country \\
email address or ORCID}
\and
\IEEEauthorblockN{3\textsuperscript{rd} Given Name Surname}
\IEEEauthorblockA{\textit{dept. name of organization (of Aff.)} \\
\textit{name of organization (of Aff.)}\\
City, Country \\
email address or ORCID}
\and
\IEEEauthorblockN{4\textsuperscript{th} Given Name Surname}
\IEEEauthorblockA{\textit{dept. name of organization (of Aff.)} \\
\textit{name of organization (of Aff.)}\\
City, Country \\
email address or ORCID}
\and
\IEEEauthorblockN{5\textsuperscript{th} Given Name Surname}
\IEEEauthorblockA{\textit{dept. name of organization (of Aff.)} \\
\textit{name of organization (of Aff.)}\\
City, Country \\
email address or ORCID}
\and
\IEEEauthorblockN{6\textsuperscript{th} Given Name Surname}
\IEEEauthorblockA{\textit{dept. name of organization (of Aff.)} \\
\textit{name of organization (of Aff.)}\\
City, Country \\
email address or ORCID}
}

\maketitle

\begin{abstract}
This document is a model and instructions for \LaTeX.
This and the IEEEtran.cls file define the components of your paper [title, text, heads, etc.]. *CRITICAL: Do Not Use Symbols, Special Characters, Footnotes, 
or Math in Paper Title or Abstract.
\end{abstract}

\begin{IEEEkeywords}
component, formatting, style, styling, insert
\end{IEEEkeywords}

\section{Introduction}
This document is a model and instructions for \LaTeX.
Please observe the conference page limits. 

\section{Metodologia}

\subsection{Coleta de Dados}

O presente estudo utilizou o conjunto de dados “\textit{Heart Rate Prediction to Monitor Stress Level}” (Shanawad, 2021), disponível na plataforma Kaggle. O dataset reúne atributos fisiológicos derivados de sinais de eletrocardiograma (ECG) obtidos em diferentes indivíduos sob distintas condições de estresse, com o objetivo de analisar a relação entre variabilidade da frequência cardíaca e níveis de estresse.

Os dados foram fornecidos em seis arquivos .csv, divididos em subconjuntos de treinamento e teste, organizados nos domínios do tempo, frequência e não linearidade, contendo variáveis como MEAN$\textunderscore$RR, RMSSD, LF/HF, SD1, Sampen, entre outras. O conjunto de treinamento, utilizado neste trabalho, apresenta $N = 369.289$ amostras e $D = 36$ variáveis preditoras, além de duas variáveis-alvo: HR (frequência cardíaca) e condition (nível de estresse).

A variável condition possui L = 3 classes, distribuídas em no stress ($54,18\%$ das amostras), interruption ($28,47\%$ das amostras) e time pressure ($17,35\%$ das amostras), indicando leve desbalanceamento entre categorias. A análise descritiva inicial foi conduzida em {\tt Python} com a biblioteca Pandas, permitindo identificar dimensões, distribuição de classes e consistência dos dados. Apenas o conjunto de treinamento foi utilizado, visto que os arquivos de teste não contêm a variável HR, inviabilizando validação supervisionada. Dessa forma, todas as análises exploratórias e procedimentos de pré-processamento foram conduzidos exclusivamente sobre o conjunto de treinamento, abrangendo os três domínios de características.


\subsection{Refinação dos dados}
A fim de otimizar o conjunto de dados para a análise, foi realizada uma etapa rigorosa de seleção de características para remover preditores redundantes ou pouco informativos, utilizando dois critérios principais.

Para mitigar o problema de multicolinearidade entre os preditores, foi adotado um método heurístico baseado na matriz de correlação de Pearson, conforme proposto por Kuhn e Johnson \cite{b8}. O algoritmo remove o número mínimo de variáveis necessárias para que todas as correlações pareadas fiquem abaixo de um limiar, que foi definido para esse estudo como $0.8$. Em cada iteração, identifica-se o par mais correlacionado, calcula-se a média das correlações de cada variável e remove-se aquela com maior média. O processo é repetido até que não restem correlações acima do limiar, reduzindo redundâncias e melhorando a estabilidade dos modelos preditivos. Para esse estudo, a variável HR não foi considerada na aplicação do algoritmo, devido ao seu uso como variável-alvo no futuro.

Em segundo lugar, foi realizada uma análise discriminante baseada nos resultados da análise univariada condicional. As variáveis que demonstraram baixo poder de separação entre as classes de estresse, ou seja, cujas distribuições se sobrepunham significativamente entre as diferentes condições, foram consideradas não informativas e também foram eliminadas.

Este processo de dupla filtragem resultou em um conjunto final e otimizado de preditores, que foi então utilizado para a subsequente Análise de Componentes Principais.


\section*{References}

Please number citations consecutively within brackets \cite{b1}. The 
sentence punctuation follows the bracket \cite{b2}. Refer simply to the reference 
number, as in \cite{b3}---do not use ``Ref. \cite{b3}'' or ``reference \cite{b3}'' except at 
the beginning of a sentence: ``Reference \cite{b3} was the first $\ldots$''

Number footnotes separately in superscripts. Place the actual footnote at 
the bottom of the column in which it was cited. Do not put footnotes in the 
abstract or reference list. Use letters for table footnotes.

Unless there are six authors or more give all authors' names; do not use 
``et al.''. Papers that have not been published, even if they have been 
submitted for publication, should be cited as ``unpublished'' \cite{b4}. Papers 
that have been accepted for publication should be cited as ``in press'' \cite{b5}. 
Capitalize only the first word in a paper title, except for proper nouns and 
element symbols.

For papers published in translation journals, please give the English 
citation first, followed by the original foreign-language citation \cite{b6}.

\begin{thebibliography}{00}
\bibitem{b1} G. Eason, B. Noble, and I. N. Sneddon, ``On certain integrals of Lipschitz-Hankel type involving products of Bessel functions,'' Phil. Trans. Roy. Soc. London, vol. A247, pp. 529--551, April 1955.
\bibitem{b2} J. Clerk Maxwell, A Treatise on Electricity and Magnetism, 3rd ed., vol. 2. Oxford: Clarendon, 1892, pp.68--73.
\bibitem{b3} I. S. Jacobs and C. P. Bean, ``Fine particles, thin films and exchange anisotropy,'' in Magnetism, vol. III, G. T. Rado and H. Suhl, Eds. New York: Academic, 1963, pp. 271--350.
\bibitem{b4} K. Elissa, ``Title of paper if known,'' unpublished.
\bibitem{b5} R. Nicole, ``Title of paper with only first word capitalized,'' J. Name Stand. Abbrev., in press.
\bibitem{b6} Y. Yorozu, M. Hirano, K. Oka, and Y. Tagawa, ``Electron spectroscopy studies on magneto-optical media and plastic substrate interface,'' IEEE Transl. J. Magn. Japan, vol. 2, pp. 740--741, August 1987 [Digests 9th Annual Conf. Magnetics Japan, p. 301, 1982].
\bibitem{b7} M. Young, The Technical Writer's Handbook. Mill Valley, CA: University Science, 1989.
\bibitem{b8} Kuhn, M., \& Johnson, K. (2018). Applied predictive modeling. Springer.
\end{thebibliography}
\vspace{12pt}
\color{red}
IEEE conference templates contain guidance text for composing and formatting conference papers. Please ensure that all template text is removed from your conference paper prior to submission to the conference. Failure to remove the template text from your paper may result in your paper not being published.

\end{document}
